\chapter{Où trouver un bon dépôt pour pas cher~?}

\section{Une petite liste}

Il existe de nombeuses entreprises ou organisations qui proposent des services/\svn{} en ligne, gratuits, payants, avec plus ou moins de possibilités, \ldots\\

\begin{itemize}
  \item \textbf{Google Code~:} \url{http://code.google.com/}~; serveurs publics uniquement
  \item \textbf{Source Forge~:} \url{http://sourceforge.net/}~; serveurs publics uniquement
  \item \textbf{Assembla~:} \url{http://www.assembla.com/}~; serveurs publics gratuits, privés payant
  \item \textbf{PC-Show~:} \url{https://svn.pc-shows.com/}~; serveurs privés gratuits avec Trac
\end{itemize}

Vous pouvez aussi vous créer votre propre serveur \svn, ce n'est pas forcément la meilleur solution, mais si vous voulez absolument l'héberger, vous trouverez toutes la doc sur \href{http://www.google.com/}{Google}.

\section{Partir du bon pied}

En fonction du site sur lequel vous allez créer votre projet, il se peut que vous ayez à créer les répertoires de bases d'un dépôt, sans quoi il vous sera plus difficile de tirer parti de certaines fonctionnalités comme les branches ou les tags.\\

Vérifiez donc toujours que vous avez les trois répertoires \texttt{branches}, \texttt{tags} et \texttt{trunk} à la racine de votre dépôt.