\chapter{Introduction}

\section{Présentation du \tp}

Nous arrivons enfin dans la partie plaisante de cette soirée~! Nous espérons que vous prendrez autant de plaisir que nous à réaliser ce jeu dont la renommée n'est plus à faire.

Allez-vous chercher un café ou une autre boison caféinée dans le distributeur le plus proche, ça va bientôt commencer.\\

Dans ce second \tp, nous vous proposons de vous guider pas-à-pas dans la réalisation d'un jeu, copie de Space Invaders dont vous pourrez voir plus loin des captures d'écrans.

\section{Création du dépôt de travail sur le serveur \svn}

Afin de mettre en application ce que vous venez d'apprendre dans le \tp{} précédent, commencez par créer un nouveau dépôt sur le serveur de votre salle.\\

Pour ce \tp, vous avez le choix entre plusieurs dépôts~:
\begin{itemize}
 \item \textbf{TP 2 2D + surcouche~:} dépôt contenant une surcouche d'abstraction. Choisissez ce dépôt si vous n'avez pas encore de connaissance dans l'utilisation de \textsc{xna}.
 \item \textbf{TP 2 2D~:} dépôt contenant uniquement la structure de base d'un nouveau projet \textsc{xna} avec les images et les sons du jeu. Choisissez ce dépôt si vous vous sentez d'attaque pour réaliser les moteurs de base (graphique, \ldots), non documentés dans ce \tp.
 \item \textbf{Vide~:} choisissez ce dépôt si vous ne voulez pas utiliser \textsc{xna} mais que l'utilisation de la surcouche est trop simple pour vous. Les sprites ainsi que les sons sont disponibles dans les fichiers en téléchargement sur le serveur de votre salle.
\end{itemize}