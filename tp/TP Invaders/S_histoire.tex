\chapter{Histoire et règles}

Pour essayer de reproduire un jeu, il est important de comprendre quelles sont
ses règles et dans quel contexte il est sorti. Cependant, si cette partie vous
ennuie, que vous connaissez déjà tout ou que vous vous en foutez, rien ne vous
empêche de la passer.

\section{Histoire}

\subsection{Version courte}

Taito sort \emph{Space Invaders} en juillet 1978 au Japon sous forme d’une borne
d’arcade et fait un véritable carton.

\subsection{Version romanesque}

Durant la première moitié des années 70, le (petit) monde du jeu vidéo ne
connait que des clones du célèbre jeu \emph{Pong} (Atari, 1972). Plusieurs
entreprises lancent leurs clones plus ou moins évolués de \emph{Pong} jusqu'à
épuisement total du concept.

Tomohiro Nishikado, trentenaire employé de Taito et chargé de
concevoir des jeux, se rend bien compte de ce manque de créativité et essaie,
comme de nombreux autres, de complexifier le concept de Pong (\emph{Soccer} en
1973, \emph{Speed Race} en 1975 \dots). Mais la sortie de \emph{Breakout} en
1976 par Atari sera un vrai choc pour lui et lui fera prendre conscience que
la clé du succès est dans la simplicité. Pour re-situer, ce jeu est le premier
casse-brique~: le joueur dirige une plateforme latéralement en bas de l'écran
et doit, pour gagner des points, viser un véritable mur de briques en haut de
l'écran en influant avec sa plateforme sur une bille qui rebondit à l'infini.

Il étudie alors longuement \emph{Breakout} et s'en sert comme base pour son
nouveau projet. La première étape est de remplacer la bille par un tir droit,
les briques par des cibles et de les mettre en mouvement pour avoir un
minimum de challenge.

Mais le jeu obtenu n'est toujours pas intéressant. C'est alors que Nishikado a
une idée révolutionnaire pour l'époque~: les cibles pourront elles aussi
attaquer~! Certains considèrent ainsi qu'il s'agit du premier jeu où la machine
peux attaquer le joueur. N'étant plus à une révolution vidéo-ludique près, le
créateur décide de changer le très classique compte à rebours pour la
limitation du temps par la descente progressive des cibles vers le joueur,
augmentant un peu plus son stress.

Le concept commence à devenir alléchant mais si on prend un peu de recul, on
se rend rapidement compte que des cibles qui attaquent n'est pas très vendeur.
Après avoir essayé de les remplacer par des tanks, des avions, des êtres
humains (et s'être fait grondé par la direction pour ça au passage), il se laisse
influencer par le succès de \emph{Star Wars} aux États-Unis et adopte des
vagues d'extra-terrestres. Au moins personne ne viendra défendre leurs droits,
enfin on espère\dots
\\

Le jeu sort en mi-juillet 1978 au Japon et fait un véritable carton. Taito
peine à faire face à la demande et va jusqu'à proposer à ses concurrents de lui
acheter une licence et de fabriquer leurs propres bornes. 300~000 bornes seront
vendues au total en ne comptant que celles de Taito, ce qui est énorme pour
l'époque. En 1979, une enquête policière au Japon sur le monde de l'arcade
révèle qu'environ 80\% des bornes du pays sont des bornes \emph{Space Invaders}~!

Les joueurs sont eux très présents et n'hésitent pas à glisser la pièce de 100
Yens requise pour lancer une partie. Les joueurs sont d'ailleurs si nombreux qu'une
véritable pénurie de pièces de 100 Yens survient au Japon à cette époque et est
référencée dans les archives de la banque nationale du Japon. Une enquête sera
lancée par le ministère de l'économie et le jeu sera cité comme cause de cette
pénurie à l'assemblée nationale japonaise~!

\section{Règles}

Vous l'aurez sans doute deviné, \emph{Space Invaders} est l'un des père du
\emph{Shoot'em up}. Des ennemis apparaissent en haut de l'écran et le
joueur doit les éliminer avec les armes dont il dispose avant que eux ne le
détruisent.

Parlons un peu plus en détail du jeu, même si vous n'êtes pas obligé de
réaliser une copie conforme de l'original. Il existe quatre types
d'ennemis~: les poulpes qui rapportent 10~points, les crabes qui rapportent
20~points, les seiches qui rapportent 30~points et enfin les soucoupes volantes
qui rapportent un nombre inconnu de points. Une vague se compose de 5~lignes,
11~colonnes, observez les captures d'écran pour voir leur disposition.

Les monstres se déplacent par paquets horizontalement et descendent d'une case
lorsqu'ils heurtent un des bords de l'écran tout en changeant de direction.
Ils tirent de manière aléatoire à intervalle régulier, en ligne droite et, la
plupart du temps, dans la zone où se trouve le joueur. La soucoupe volante est
particulière, elle n'apparaît que de temps en temps, tout en haut de l'écran,
et ne fait qu'un seul trajet (accompagné d'un son caractéristique). Si le joueur
parvient à la détruire, il gagne un nombre important de points.

Le joueur dispose d'un tir droit qui n'évolue pas et qui détruit un ennemi dès
qu'il le touche. Quatre boucliers sont présent en bas de l'écran et le
protègent des attaques ennemies. Mais ces boucliers se détruisent peu à peu et
finissent par disparaître.

Les vagues de monstres font des allers et retours à l'écran~: à chaque fois
qu'elles touchent l'un de ses bord, les monstres descendent tous d'une case.
Plus le joueur tue des monstre, plus la vague sera rapide, la plus grande
accélération se produit donc lorsqu'il ne reste plus que quelques monstres.
Si le joueur est trop nul ou ne sait pas viser, ce qui revient au même, alors
les monstres continuent leur chemin en \og effaçant \fg les boucliers et
finissent par rentrer en collision avec le joueur lui même.
\\

Le jeu original est sortie sous forme de borne d'arcade, il y a donc un
système de crédits. Lorsqu'un joueur insère une pièce, il obtient un crédit
qui lui donne droit à une partie et 3 vies par défaut. Le joueur n'a pas de
santé~: s'il est touché, il perd une vie et la partie continue sans être remise
à zéro. Lorsqu'il meurt et ne possède plus de vie, c’est \verb+Game Over+ et
son score est enregistré s'il est supérieur au meilleur score actuel. À
cette époque, les \emph{Continue} n'étaient pas encore nés.

Il existe un mode deux joueurs, il faut posséder deux crédits et appuyer
sur le bouton \og deux joueurs \fg, le jeu est tout à fait normal mais une
fois qu'un joueur perd une vie, c'est à l'autre de jouer. Les deux joueurs
utilisent les mêmes touches et jouent à tour de rôle. Chacun a son propre
terrain de jeu et sa propre configuration de monstres (lesquels ont été
détruits, etc.) et elle est restituée à chaque changement de tour.
